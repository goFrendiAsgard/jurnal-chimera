\documentclass[conference]{IEEEtran}
\usepackage{blindtext, graphicx}
\usepackage{amsmath}
\usepackage{listings}


\ifCLASSINFOpdf
\else
\fi


% correct bad hyphenation here
\hyphenation{op-tical net-works semi-conduc-tor}


\begin{document}

\title{Chimera: Agnostic Language Component Based Framework using NodeJS and CLI}

\author{\IEEEauthorblockN{Go Frendi Gunawan}
\IEEEauthorblockA{STIKI Malang\\ Malang, Indonesia\\
Email: frendi@stiki.ac.id}
\and
\IEEEauthorblockN{Mukhlis Amien}
\IEEEauthorblockA{STIKI Malang\\
Malang, Indonesia\\
Email: amien@stiki.ac.id}
\and
\IEEEauthorblockN{Jozua Ferjanus Palandi}
\IEEEauthorblockA{STIKI Malang\\
Malang, Indonesia\\
Email: jozuafp@stiki.ac.id}}

% make the title area
\maketitle


\begin{abstract}
%\boldmath
Component Based Software Engineering (CBSE) is a branch of software 
engineering that emphasizes the separation of concerns with respect to the 
wide-ranging functionality available throughout a given software system.  The 
main advantage of CBSE is separation of components. A single component will 
only focus on a single task or related collection of tasks. Allowing software 
developer to reuse the component for other use-cases. By using this approach, 
software developer doesn't need to deal with spaghetti code. Several 
approaches has been developed in order to achieve ideal CBSE. The earliest 
implementation was UNIX pipe and redirect, while the newer approach including 
CORBA, XML-RPC, and REST. Our framework, Chimera, was built on top of Node JS. 
Chimera allows developer to build pipe flow in a chain (a YAML formatted file) 
as well as defining global variables. Compared to UNIX named and unnamed pipe, 
this format is easier and more flexible. On the other hand, unlike XML-RPC, 
REST, and CORBA, chimera doesn't enforce users to use special protocol such as 
HTTP (except for distributed computing scenario). Nor it require the components
to be aware that they works on top of the framework.
\end{abstract}

% Note that keywords are not normally used for peerreview papers.
\begin{IEEEkeywords}
Chimera, Language Agnostic, Component-Based Software Engineering, CBSE, Node JS, CLI.
\end{IEEEkeywords}

\IEEEpeerreviewmaketitle

\section{Introduction}

Component based software development approach is based on the idea to develop 
software systems by selecting appropriate off-the shelf components and then to 
assemble them with a well-defined software architecture \cite{kaur2010component}.

In order to implement component-based software engineering (CBSE), several 
approaches has beeen performed. The earliest attempt was UNIX pipe mechanism 
\cite{mcilroy1968mass}. Pipe mechanism was not the only attempt to achieve CBSE.
The more modern approaches including XML-RPC \cite{xmlrpc} and JSON-RPC \cite{jsonrpc}. 
Later, Object Management Group (OMG) introduced a new standard named CORBA (Common
Object Request Broker Architecture) \cite{corba}. Another interesting approach was 
introduced by Two Sigma Open Source. Two Sigma created a platform known as Beaker
Notebook \cite{beakernotebook}. Beaker Notebook is mainly used for research purpose. 
On 2016, Feilhauer and Sobotka introduce another platform called DEF 
\cite{feilhauer2016def}.

Aside from Unix Pipe, all other mechanism require the components to be aware that 
they are part of the framework. This means that you cannot use old programs (e.g:
\lit{cal} and \lit{cowsay}) as XML-RPC or CORBA component. At least additional layer
and adjustment has to be built.

CORBA, XML-RPC, and JSON-RPC also needs HTTP protocol since they were designed for 
in client-server architecture. It imply that you need to build a web server in order
to use the mechanisms. However, in any use case that only need a single computer,
this is not ideal.

Considering the advantages and disadvantages of those early approaches, in this paper, 
we introduce a new CBSE framework named Chimera. This framework is much simpler since
HTTP is only required for distributed computation. Chimera also use CLI mechanism that
works in almost all OS and most programming language.
The only dependency of Chimera are NodeJS and several NPM packages.

\section{Previous Research}

In this section we will have an indepth discussion about UNIX Pipe, XML-RPC, CORBA, and
DEF.


\subsection{UNIX Pipe}

The very first implementation of CBSE was UNIX pipe mechanism \cite{mcilroy1968mass}. 
UNIX pipe allows engineer to pass output of a single program as an input of 
another program. Since a lot of server is UNIX or linux based, this pipe 
mehanism availability is very high. Even DOS also provide similar mechanism 
\cite{dos7command}.

Beside of it's high availability, at some point, UNIX pipe also support parallel
processing through named-pipe mechanism. The named-pipe mechanism can be used to 
provide cheap parallel processing \cite{conway2003parallel}. 

Although pipe mechanism provide high availability and capability, 
it has several limitations. For example, it needs external file as temporary 
container. The external file has to be deleted once the operation performed. 
This approach is not straight forward, thus, some efforts is needed in order to 
to build a working named-pipe based computation. 

By design, pipe mechanism assume a program's standard output turns into another 
program's standard output. For simple use cases, this might be the best approach. 
However, at some point, when the program become more complicated, memory sharing 
and network access might be needed. Using a mere pipe mechanism to support those 
requirement is either hard or impossible.


\subsection{XML-RPC and JSON-RPC}

XML-RPC is a spec and a set of implementations that allow software running on 
disparate operating systems, running in different environments to make procedure 
calls over the Internet. XML-RPC using HTTP as the transport and XML as the encoding. It is designed to be as simple as possible, while allowing complex data structures to be transmitted, processed and returned \cite{xmlrpc}.

JSON-RPC is lightweight remote procedure call protocol similar to XML-RPC 
\cite{jsonrpc}. The main difference between XML-RPC and JSON-RPC is the data transfer
format. In most cases, JSON is more lightweight compared to XML.

XML-RPC and JSON-RPC are heavily depend on HTTP for inter-process-communication 
protocol. This is ideal for client-server architecture as HTTP is quite common and
easy to be implemented.

As implied by their name, XML-RPC and JSON-RPC are basically another implementation
of RPC (Remote Procedure Call). Implementing XML or JSON-RPC means that you can 
separate your program's component and put them accross different computers. On the
other hand, your components can be shared among different programs. The sensible
code separation and reusability is the advantage of XML and JSON RPC.


\subsection{SOAP}

SOAP stands for Simple Object Access Protocol. SOAP is a lightweight protocol 
intended for exchanging structured information in a decentralized, distributed 
environment \cite{soap}. SOAP was built on top of XML-RPC. It uses XML format as 
well as HTTP protocol.

SOAP was built with Object Oriented Programming in mind. At that time, OOP is
quite popular.

\subsection{CORBA}

The Common Object Request Broker Architecture is a standard defined by the Object 
Management Group designed to facilitate the communication of systems that are 
deployed on diverse platforms. CORBA enables collaboration between systems on 
different operating systems, programming languages, and computing hardware 
\cite{corba}. 

CORBA was built by OMG, which is also responsible for creating UML standard.


\subsection{REST}
\blindtext

\subsection{DEF}

DEF - A programming language agnostic framework and execution environment 
for the parallel execution of library routines \cite{feilhauer2016def}. 
DEF focus on parallel processing by enabling shared memory and message passing. 
DEF needs several components, using JSON as data exchange format. 
Compared to CORBA, Matlab, and Parallel Fortran, DEF is better in term of 
parallelism and language agnosticism. CORBA for example, doesn't support matlab and 
octave \cite{feilhauer2016def}. 

However, DEF still depend on REST for communication. Consequently, in order to 
build DEF architecture, a web server is needed. Also, each component should 
follow certain rules in order to be able to communicate in DEF protocol.

\subsection{Beaker Notebook}

Beaker Notebook \cite{beakernotebook} is also considered as an interesting 
approach of CBSE. The platform was developed by Two Sigma Open Source and 
mainly used for research use. Like DEF, beaker has a shared storage containing 
global variables. The global variables is accessible by any pieces of program 
in a certain notebook. However, extra works is needed so that a programming 
language is available in a notebook.

\section{Our Approach}

Based on previous approaches, it seems that Beaker Notebook and DEF is better 
in terms of language agnoticism and shared memory. However, UNIX pipe is 
better in terms of availability and architecture independence.

\section{Conclusion}
Chimera 

%\appendices
%\section{Proof of the First Zonklar Equation}

% use section* for acknowledgement
\section*{Acknowledgment}
The authors would like to thank...

% Can use something like this to put references on a page
% by themselves when using endfloat and the captionsoff option.
\ifCLASSOPTIONcaptionsoff
  \newpage
\fi

\bibliographystyle{IEEEtran}
\bibliography{./citation}

% that's all folks
\end{document}

