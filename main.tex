\documentclass[conference]{IEEEtran}
\usepackage{blindtext, graphicx}
\usepackage{amsmath}
\usepackage{listings}


\ifCLASSINFOpdf
\else
\fi


% correct bad hyphenation here
\hyphenation{op-tical net-works semi-conduc-tor}


\begin{document}

\title{Chimera: Agnostic Language Component Based Framework using NodeJS and CLI}

\author{\IEEEauthorblockN{Go Frendi Gunawan}
\IEEEauthorblockA{STIKI Malang\\
Malang, Indonesia\\
Email: frendi@stiki.ac.id}
\and
\IEEEauthorblockN{Mukhlis Amien}
\IEEEauthorblockA{STIKI Malang\\
Malang, Indonesia\\
Email: amien@stiki.ac.id}
\and
\IEEEauthorblockN{Jozua Ferjanus Palandi}
\IEEEauthorblockA{STIKI Malang\\
Malang, Indonesia\\
Email: jozuafp@stiki.ac.id}}

% make the title area
\maketitle


\begin{abstract}
%\boldmath
    Component Based Software Engineering (CBSE) is a branch of software engineering that emphasizes the separation of concerns with respect to the wide-ranging functionality available throughout a given software system.  The main advantage of CBSE is separation of components. A single component will only focus on a single task or related collection of tasks. Allowing software developer to reuse the component for other use-cases. By using this approach, software developer doesn't need to deal with spaghetti code. Several approaches has been developed in order to achieve ideal CBSE. The earliest implementation was unix pipe and redirect, while the newer approach including CORBA, XML-RPC, and REST. Our framework, Chimera, was built on top of Node JS. Chimera allows developer to build pipe flow in a chain (a YAML formatted file) as well as defining global variables. Compared to unix named and unnamed pipe, this format is easier and more flexible. On the other hand, unlike XML-RPC, REST, and CORBA, chimera doesn't enforce users to use http protocol.
\end{abstract}

% Note that keywords are not normally used for peerreview papers.
\begin{IEEEkeywords}
Chimera, Language Agnostic, Component-Based Software Engineering, Node JS, CLI.
\end{IEEEkeywords}

\IEEEpeerreviewmaketitle



\section{Introduction}
Component based software development approach is based on
the idea to develop software systems by selecting appropriate
off-the shelf components and then to assemble them with a well-
defined software architecture. \cite{kaur2010component}


\subsection{Subsection Heading Here}
\blindtext


\section{Conclusion}
\blindtext


\appendices
\section{Proof of the First Zonklar Equation}
\blindtext

% use section* for acknowledgement
\section*{Acknowledgment}
The authors would like to thank...

% Can use something like this to put references on a page
% by themselves when using endfloat and the captionsoff option.
\ifCLASSOPTIONcaptionsoff
  \newpage
\fi

\bibliographystyle{IEEEtran}
\bibliography{IEEEabrv,./main}

\begin{thebibliography}{1}

\bibitem{IEEEhowto:kopka}
H.~Kopka and P.~W. Daly, \emph{A Guide to \LaTeX}, 3rd~ed.\hskip 1em plus
  0.5em minus 0.4em\relax Harlow, England: Addison-Wesley, 1999.

\end{thebibliography}



% that's all folks
\end{document}

